\documentclass[12pt,a4paper]{beamer}
\usepackage[utf8x]{inputenc}
\usepackage{ucs}
\usepackage[english]{babel}
\usepackage{amsmath}
\usepackage{amsfonts}
\usepackage{amssymb}
\author{Michael Haas <haas@cl.uni-heidelberg.de>}
\title{Kernelized Locality-Sensitive Hashing}
\subtitle{Seminar Learning Similarity Metrics, Stefan Riezler, Artem Sokolov}
\date{21-05-2013}
\begin{document}


\begin{frame}
\maketitle
\end{frame}

\begin{frame}{Overview}
\begin{itemize}
\item Problem Definition \& Motivation
\item Related Work
\item Locality-Sensitive Hashing
\item Kernelized Locality-Sensitive Hashing
\item Results
\end{itemize}
\end{frame}

\begin{frame}{Problem}
\begin{itemize}
\item Given a document D, find similar documents in set S.
\item \textit{For s in S: if sim(D,S) then yield s}
\item $|S| = 10000$ - easy
\item $|S| = 15 \times 10^9$ - not so much
\item Problem amplified: curse of dimensionality
\item Need \textbf{sublinear} performance
\end{itemize}
\end{frame}

\begin{frame}{Related work}
\begin{itemize}
\item k-d trees, using spatial partitions
    \begin{itemize}
    \item Still worst-case linear query time
    \end{itemize}
\item Tree-based search structures
    \begin{itemize}
    \item Can also degenerate to linear time
    \end{itemize}
\end{itemize}
\end{frame}

\begin{frame}{Basic Idea}
\begin{itemize}
\item \textbf{Approximate} similarity search
\item Trade-Off: \textbf{predictable} accuracy loss VS fast queries
\item Locality-Sensitive Hashing: high probably of collision for similar items
\item Using sketches and subset of the data
\end{itemize}
\end{frame}

\begin{frame}{Overview}
\begin{itemize}
\item Problem Definition \& Motivation
\item Related Work
\item \textbf{Locality-Sensitive Hashing}
\item Kernelized Locality-Sensitive Hashing
\item Results
\end{itemize}
\end{frame}





\begin{frame}{Locality-Sensitive Hashing}
\begin{itemize}
\item Map each data point to \textit{b}-bit vector
\item \textit{b}-bit Vector: apply $h_1,h_2,..,h_n$ hash-functions to objects
\item Locality-Sensitive Hashing Property \cite{lsh}: $$Pr[h(x_i)=h(x_j)]=sim(x_i,x_j)$$
\item $sim(x,y) \in [0,1]$ is similarity function
\end{itemize}
\end{frame}

\begin{frame}{LSH: Runtime}
\begin{itemize}
\item Query time to find ($1+\epsilon$) nearest neighbors: $O(n^{1/1+\epsilon})$
\item Sublinear for $\epsilon>0$
\end{itemize}
\end{frame}


\begin{frame}{LSH: Finding Hash functions}
\begin{itemize}
\item Where do we get hash functions?
\item Idea: Random Hyperplanes $r$!
$$h_r(x)= \begin{array}{l l}
    1 & \quad \text{if} r^Tx \geq 0\\
    0 & \quad \text{otherwise}
  \end{array} .$$
\end{itemize}
\end{frame}

\begin{frame}{LSH: Locality-Sensitive Hashing property}
\begin{itemize}
\item Hash function $h_r$ satifies Locality-Sensitive Hashing property $Pr[h(x_i)=h(x_j)]=sim(x_i,x_j)$
\item Goemans and Williamson showed for random vector $r$:
$$Pr[sign(x^T_{i}=sign(x^T_{j})]=1-\frac{1}{\pi}\cos^{-1}\frac{x^{T}_{i} x_{j}}{||x_{i}| ||x_{j}||}$$
\end{itemize}
\end{frame}



\begin{frame}{LSH: Algorithm Idea}
\begin{itemize}
\item Draw \textit{b} random vectors $r$ from $N(0,I)$
\item Build hash vector using \textit{b} hash functions
\item Hash table consists of hash keys and pointers to data item
\item Given incoming query q, hash q using same \textit{b} hash functions
\item Hash bucket for q contains nearest neighbor candidates
\end{itemize}
\end{frame}

\begin{frame}{LSH: Algorithm}
\begin{itemize}
\item Given $n$ database keys, form $M=2n^{1/(1-\epsilon)}$ permutations of keys
\item Sort M lists of permuted hash keys
\item Compute query hash key q, look up in M lists via binary-search
\item 2M nearest examples are nearest neighbor candidates
\item $O(n^{1/1+\epsilon})$ database items considered
\end{itemize}
\end{frame}

\begin{frame}{Overview}
\begin{itemize}
\item Problem Definition \& Motivation
\item Related Work
\item Locality-Sensitive Hashing
\item \textbf{Kernelized Locality-Sensitive Hashing}
\item Results
\end{itemize}
\end{frame}

\begin{frame}{Kernelized Locality-Sensitive Hashing}
\begin{itemize}
\item Kernel functions can be seen as similarity measure: $sim(x,y) = k(x,y) = \phi(x_{i})^{T}\phi(x_{j}) $
\item Using kernel functions in LSH is easy then: build $h_r(x)$ by sampling vector $r$ from kernel space
$$h_r(x)=sign(r^T\phi(x))$$
\item Not all kernel functions have known embeddings! \cite{klsh}
\item Not all known embeddedings are easily computable!
\item Example: Feature space of the Radial Basis Kernel has infinite number of dimensions
\end{itemize}
\end{frame}

\begin{frame}{KLSH: Idea}
\begin{itemize}
\item Task: Find $r$ so that $r^T\phi (x)$ can be computed via kernel function
\item Idea: Construct $r$ as subset of weighted database items
\item $r$ must be Gaussian
\end{itemize}
\end{frame}

\begin{frame}{KLSH: }
\begin{itemize}
\item Consider each point $\phi(x)$ as vector from underlying distribution D
\item $\mu$ and $\Sigma$ unknown
\item $z_{t} = \frac{1}{t}\sum_{i \in S}{\phi (x_{i})}$
\item For sufficiently large $t$ (CLT):
$$ \tilde{z}_{t} = \sqrt{t}(z_{t} - \mu) $$
\item $ \tilde{z}_{t}$ distributed according to  $N(0,\Sigma)$
\item $\Sigma^{-1/2}\tilde{z}_{t}$ distributed according to $N(0,I)$ (whitening transform: entries now uncorrelated and with variance 1)
$$h_r(\phi ( x) )= \begin{array}{l l}
    1 & \quad \text{if} \phi (x)^{T} \Sigma^{\frac{-1}{2}}\tilde{z}_{t} \geq 0\\
    0 & \quad \text{otherwise}
  \end{array}$$
\end{itemize}
\end{frame}



\begin{frame}{Estimating mean and covariance}
\begin{itemize}
\item But: $\Sigma$ and $\mu$ are unknown
\item Approximate by sampling the data! Choose $p$ database items
\item Implicitly:
    \begin{itemize}
    \item $\mu = \frac{1}{p}\sum_{i=1}^{p}\phi(x)$
    \item Similar for $\Sigma$
    \end{itemize}
\end{itemize}
\end{frame}


\begin{frame}
\begin{itemize}
\item $K$: Kernel matrix over sampled points $p$
\item Eigendecomposition: $K=U \Theta U^{T}$
\item Eigendecomposition: $ \Sigma = V \Lambda V^{T} $, $ \Sigma^{-1/2} = V \Lambda^{-1/2} V^{T}$
\item $h(\phi(x)) = sign(\phi(x)^T V \Lambda^{-1/2} V^{T} \tilde z_t{t})$
\item $v_{k}$: k-th Eigenvector of $\Sigma$
\item $u_{k}$: k-th Eigenvector of $K$
\end{itemize}
\end{frame}

\begin{frame}
\begin{itemize}
\item According to derivation of kPCA:
$$ v_{k}^{T} \phi (x) = \sum_{i=1}^{p}  \frac{1}{ \sqrt{ \theta_{k} } } u_{k}(i) \phi (x_{i})^{T} \phi (x)  $$
\item We also get: $ \phi(x)^{T} V \Lambda^{-1/2} V^{T} \tilde z_{t} = \sum_{k=1}^{p} \frac{1}{\sqrt{\theta_{k}} v_{k}^{T} \phi(x) v_{k}^{T} \tilde z_{t} }$
\item Substitute first equation for eigenvector inner products
$$ \phi(x)^{T} V \Lambda^{-1/2} V^{T} \tilde z_{t} = $$
$$\sum_{k=1}^{p} \frac{1}{\sqrt{\theta_{k}}} 
(\sum_{i=1}^{p} \frac{1}{ \sqrt{\theta_{k}} } u_{k}(i) \phi (x_{i})^{T} \phi (x) )
(\sum_{i=1}^{p} \frac{1}{ \sqrt{\theta_{k}} } u_{k}(i) \phi (x_{i})^{T} \tilde z_{t}  )
$$
\end{itemize}
\end{frame}

\begin{frame}
\begin{itemize}
\item (..) further derivations
$$ = \sum_{i=1}^{p}\sum_{j=1}^{p} K_{ij}^{-3/2}(\phi ( x_{i}^{T} \phi (x) ) (  x_{i}^{T} \tilde z_{t}  ) $$
$$ = \sum_{i=1}^{p} w(i) (  x_{i}^{T} \phi (x) ) $$
\item where $w(i) = \sum_{j=1} K_{ij}^{-3/2} (  x_{i}^{T} \tilde z_{t}) $
\item Recall $\tilde z_{t} = \frac{1} {\sqrt{t}} \sum_{i \in S}{\phi (xi)} $
$$w(i) = \frac{1}{\sqrt{t}}\sum_{j=1}^{p}\sum_{t \in S} K_{ij}^{-3/2} K_{jl} $$
\end{itemize}
\end{frame}





\begin{frame}{KLSH: Simplification of w }
\begin{itemize}
\item $t$ points are subset of $p$ sampled points
\item $e_s$ is vector with ones in entries corresponding to indices of S
\item $w = K^{(-1/2)} e_s$
\end{itemize}
\end{frame}



\begin{frame}{KLSH Final hash function}
\begin{itemize}
\item Effectively, $r$ needed for $h_r(x)=sign(r^T\phi(x))$ is now weighted sum of feature vectors from $p$ sampled database items
$$h(\phi(x)) = sign(\sum_{i=1}^{p}w(i)\phi(x_{i})^{T}\phi(x))$$
\item Since $k(x,y) = \phi(x_{i}^{T}\phi(x_{j}))$, we get
$$h(\phi(x)) = sign(\sum_{i=1}^{p}w(i)k(x,x_{i}))$$
\end{itemize}
\end{frame}

\begin{frame}{(K)-LSH: Algorithm}
\begin{itemize}
\item Select p data points and build kernel matrix $K$
\item Form hash table over database items:
\begin{itemize}
\item For each hash function, form $w$ ($e_s$) by sampling t indices from $[1,...,p]]$.
\item Assign bits according to $h(\phi (x)) = sign (\sum_{i} w(i) k(x, x_{i}) )$
\end{itemize}
\item Use regular LSH algorithm with $h(x)$ hash function

    \begin{itemize}
    \item Given $n$ database keys, form $M=2n^{1/(1-\epsilon)}$ permutations of keys
    \item Sort M lists of permuted hash keys
    \item Compute query hash key q, look up in M lists via binary-search
    \item 2M nearest examples are nearest neighbor candidates
    \end{itemize}

\end{itemize}
\end{frame}

\begin{frame}{Runtime}
\begin{itemize}
\item Computing $K^{-1/2}: O(p^{3})$ - Offline
\item Each hash function: $O(p^{2})$ to compute $w$ - Offline
\item Each hash function function evaluation: $O(p)$
\item $p = O(\sqrt{n})$ guarantees sublinear time
\end{itemize}
\end{frame}

\begin{frame}{Overview}
\begin{itemize}
\item Problem Definition \& Motivation
\item Related Work
\item Locality-Sensitive Hashing
\item Kernelized Locality-Sensitive Hashing
\item \textbf{Results}
\end{itemize}
\end{frame}


\begin{frame}{References}
\begin{thebibliography}{-}
\bibitem{klsh} Kulis, B., Grauman, K.: Kernelized Locality-Sensitive Hashing for Scalable Image Search.
In Proc. 12th International Conference on Computer Vision (ICCV), 2009.
\bibitem{lsh} Charikar, Moses S.: Similarity estimation techniques from rounding algorithms. Proceedings of the thiry-fourth annual ACM symposium on Theory of computing. ACM, 2002.
\end{thebibliography}
\end{frame}
\end{document}
